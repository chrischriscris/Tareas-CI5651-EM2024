1. (3 puntos) — Considere el algoritmo StrupidSort para ordenar un arreglo de menor a
mayor:
def stupidSort(a):
for x in permutaciones(a):
if ordenado(x):
return x
Puede suponer que permutaciones es un iterador que devuelve, una a una, todas las permutaciones de un arreglo en tiempo constante. Además, ordenado es un predicado que
verifica si un arreglo está ordenado de menor a mayor en tiempo Θ(n).
Diga cotas inferior y superior para stupidSort (correspondiendo al peor y mejor caso,
respectivamente). Esto es, defina dos funciones f y g tal que se cumpla que:
• stupidSort ∈ O(f) • stupidSort ∈ Ω(g)
2. (3 puntos) — En la sede del banco caben hasta C personas. La sede empieza vacía y los
clientes entran uno a uno. Cuando la capacidad de la sede es superada, la misma cierra y
la totalidad de los clientes es transferida a una sede más grande, con capacidad 2 × C.
Nótese que si contamos cuántas personas deben trasladarse al entrar un cliente a la sede,
ocurre uno de dos casos:
• Si aún no se ha superado la capacidad de la sede, la cantidad de personas que debe
trasladarse es O(1) — únicamente el cliente que entra.
• Si se supera la capacidad de la sede, la cantidad de personas que debe trasladarse es
O(C) — todos los clientes en la sede del banco más el cliente nuevo.
Demuestre que el orden amortizado, en el peor caso, para el ingreso de un cliente es O(1)
3. (3 puntos) — El problema ⊕–SAT modificación sobre el problema de satisfacibilidad booleana,
donde la fórmula proposicional está en forma normal conjuntiva, pero cada cláusula está
separada por disyunciones exclusivas (en vez de disyunciones tradicionales).
Recordemos que P ⊕ Q es cierto sí y solo sí uno de entre P y Q es cierto (el otro siendo
falso).
Diga si ⊕–SAT ∈ P o si ⊕–SAT es NP–completo. Demuestre su afirmación.
