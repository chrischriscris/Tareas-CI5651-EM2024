\documentclass[letterpaper, 12pt]{article}
\usepackage[top=2 cm, bottom=2 cm,left=1.5 cm, right=1.5 cm]{geometry}

\usepackage[spanish,es-nodecimaldot]{babel}
\usepackage[utf8]{inputenc}
\usepackage{csquotes}

\setlength{\parindent}{1cm}
\pagestyle{empty}

\usepackage[colorlinks=true, urlcolor=blue, linkcolor=red]{hyperref} % Para insertar hipervínculos
\usepackage{amsmath}
\usepackage[outputdir=aux]{minted}
\usepackage{amssymb}
\usepackage{multicol}
\usepackage{enumerate}
\usepackage{graphicx} % Incluir imágenes
\usepackage[document]{ragged2e}
\usepackage{mathrsfs}
\usepackage[T1]{fontenc}
\usepackage[table,xcdraw]{xcolor}
\usepackage{float} %required for the placement specifier H
\makeatletter
\makeatother
% -----------------------------------------------

% Preámbulo

\title{Tarea 4: Programación dinámica}
\author{Christopher Gómez}
\date{Enero 2024}

\begin{document}
\parbox[t]{.5\linewidth}{
    \centering
    \includegraphics[scale=0.4]{logo.png}
    \begin{center}
        UNIVERSIDAD SIMÓN BOLÍVAR \\
        CI-5651 - Diseño de Algoritmos I \\
        Prof. Ricardo Monascal \\
    \end{center}
}
\hfill \framebox[5.5cm][c]{
        \parbox[t]{.8\linewidth}{
        \centering
       Resuelto por:\\ Christopher Gómez
        }}

\phantom{This text will be invisible} \\
\centerline {\textbf{Tarea 4: Programación dinámica}}
\justify

\begin{enumerate}

% -------- PREGUNTA 1 --------

\item Considere el algoritmo de \emph{Programación Dinámica} propuesto en clase para el problema de distancia de edición entre dos cadenas de caracteres.

Construya la tabla correspondiente al proceso para decidir la distancia de edición entre su \textbf{primer nombre} y su \textbf{primer apellido}, \emph{sin ahorro de memoria}.

Por ejemplo, si su nombre es \emph{Fulano Mengano Apellidez Surnameson}, se desea que construya
la tabla $6 \times 9$ para la distancia de edición entre \texttt{fulano} y \texttt{apellidez}.

\begin{center}
    \begin{tabular}{|c|c|c|c|c|c|c|}
        \hline
        \rowcolor[HTML]{C0C0C0}
        \textbf{} & \textbf{} & \textbf{g} & \textbf{o} & \textbf{m} & \textbf{e} & \textbf{z} \\ \hline \rowcolor[HTML]{C0C0C0}
        \textbf{} & \cellcolor[HTML]{FFFFFF} & 1 & 2 & 3 & 4 & 5 \\ \hline \rowcolor[HTML]{C0C0C0}
        \textbf{c} & 1 & 1 & 1 & 2 & 3 & 4 \\ \hline \rowcolor[HTML]{C0C0C0}
        \textbf{h} & 2 & 2 & 2 & 2 & 3 & 4 \\ \hline \rowcolor[HTML]{C0C0C0}
        \textbf{r} & 3 & 3 & 3 & 3 & 3 & 4 \\ \hline \rowcolor[HTML]{C0C0C0}
        \textbf{i} & 4 & 4 & 4 & 4 & 4 & 4 \\ \hline \rowcolor[HTML]{C0C0C0}
        \textbf{s} & 5 & 5 & 5 & 5 & 5 & 5 \\ \hline \rowcolor[HTML]{C0C0C0}
        \textbf{t} & 6 & 6 & 6 & 6 & 6 & 6 \\ \hline \rowcolor[HTML]{C0C0C0}
        \textbf{o} & 7 & 7 & 7 & 7 & 7 & 7 \\ \hline \rowcolor[HTML]{C0C0C0}
        \textbf{p} & 8 & 8 & 8 & 8 & 8 & 8 \\ \hline \rowcolor[HTML]{C0C0C0}
        \textbf{h} & 9 & 9 & 9 & 9 & 9 & 9 \\ \hline \rowcolor[HTML]{C0C0C0}
        \textbf{e} & 10 & 10 & 10 & 10 & 10 & 10 \\ \hline \rowcolor[HTML]{C0C0C0}
        \textbf{r} & 11 & 11 & 11 & 11 & 11 & 11 \\ \hline \rowcolor[HTML]{C0C0C0}
    \end{tabular}
\end{center}

% -------- PREGUNTA 2 --------

\item Sea $A[1..n]$ un arreglo de enteros.

Decimos que $B$ es un sub-arreglo de $A$ si se pueden remover elementos del arreglo $A$,
respetando el orden en el que aparecen, para obtener $B$.

Decimos que $B[1..k]$ es bueno si el arreglo no está vacío y para todo $i$, tal que $1 \leq i \leq k$ se
cumple que $B[i]$ es divisible entre $i$.

Consideramos que dos sub-arreglos son diferentes si provienen de posiciones diferentes en $A$, incluso si los valores son iguales.

Por ejemplo, si $A = [2, 2, 1, 22, 15]$

Los sub-arreglos buenos serían 13, que son:

\begin{multicols}{5}
    \begin{itemize}
        \item $[2]$
        \item $[2, 2]$
        \item $[2, 2, 15]$
        \item $[2, 22]$
        \item $[2, 22, 15]$
        \item $[2]$
        \item $[2, 22]$
        \item $[2, 22, 15]$
        \item $[1]$
        \item $[1, 22]$
        \item $[1, 22, 15]$
        \item $[22]$
        \item $[15]$
    \end{itemize}
\end{multicols}

Queremos saber la cantidad de sub-arreglos buenos de $A$.

Se desea que diseñe un algoritmo usando \emph{Programación Dinámica}, que resuelva este problema en tiempo $O(n^2)$ y con memoria adicional $O(n)$.

% -------- PREGUNTA 3 --------

\item Se desea que implemente, en el lenguaje de su elección, un cliente para probar el proceso de inicialización virtual de arreglos. Su programa debe cumplir con las siguientes características:

\begin{itemize}
    \item  Al invocarse, debe recibir como argumento del sistema el tamaño del arreglo a utilizar.
    \item El arreglo será indexado a cero (las posiciones válidas, para un arreglo de tamaño n, serán desde la $0$ hasta la $n - 1$).

    Debe presentar al usuario un cliente con cuatro opciones:
    \begin{itemize}
        \item \texttt{ASIGNAR POS VAL}, que tiene el efecto de asignar el valor \texttt{VAL} en la posición {POS} del arreglo.

        Su programa debe reportar un error si la posición \texttt{POS} no es una posición válida del arreglo.

        \item \texttt{CONSULTAR POS}, que debe reportar si la posición \texttt{POS} está inicializada o no. En caso de estar inicializada, debe devolver el valor asociado a esa posición.

        Su programa debe reportar un error si la posición \texttt{POS} no es una posición válida del arreglo.

        \item \texttt{LIMPIAR}, que tiene el efecto de limpiar la tabla y hace que \textbf{todas} las posiciones queden sin inicializar.

        \item \texttt{SALIR}, que dale del programa.
    \end{itemize}

    \item Todas las operaciones involucradas en su programa deben tomar tiempo $\Theta(1)$.

    \item Debe implementar estas operaciones siguiendo el proceso de inicialización virtual visto en clase, usando los dos arreglos auxiliares, \textbf{no} con métodos alternativos (como tablas de hash o tipos conjuntos de la librerías de su lenguaje de escogencia).
\end{itemize}

Una implementación de este cliente en C se puede encontrar \href{https://www.google.com}{aquí}.

% -------- PREGUNTA 4 --------

\item ¡Esta aerolinea es un desastre! El transporte automatizado de equipaje se ha descompuesto y ha dejado las maletas de los pasajeros regadas por toda la pista. Es tu trabajo volver a recogerlas todas y colocarlas en el avión. Pero, ¡de prisa! Mientras más tiempo tomes, más se retrasará el vuelo en salir.

Decides hacer una lista de todas las cosas que sabes, para organizarte mejor:

\begin{itemize}
    \item La cantidad, $n$, de equipajes que se han caído. También sabes que $1 \leq n \leq 24$.

    \item La posición $(x, y)$ de cada equipaje. Además, dadas las dimensiones del aeropuerto, sabes que $|x| \leq 100$ y $|y| \leq 100$ y que ambas dimensiones son números enteros.
    \item El avión está en la posición $(0, 0)$ y es a donde quieres llevar todas las maletas que se han dispersado.
    \item Transitar entre dos puntos $a$ y $b$, toma tanto tiempo como \textbf{el cuadrado de la distancia cartesiana} entre ellos.
    \item Solamente te puedes detener en la posición de una maleta o en la del avión, por temor a que te multen las autoridades del aeropuerto.
    \item Puedes cargar a lo sumo dos maletas a la vez (una en cada mano) y una vez que tomas una maleta, solamente puedes soltarla en el avión.
\end{itemize}

¿Cuál es la mínima cantidad de tiempo necesaria para recoger todas las maletas?

Se desea que diseñe un algoritmo usando \emph{Programación Dinámica}, que resuelva este problema en tiempo y memoria $O(n \times 2^n)$.

\end{enumerate} \vspace{4mm}

\end{document}
